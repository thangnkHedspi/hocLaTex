\documentclass[12pt,a4paper]{article}
\usepackage[utf8]{vietnam}
\usepackage{amsmath}
\usepackage{amsfonts}
\usepackage{amssymb}
\usepackage{graphicx}
\usepackage[left=2cm,right=2cm,top=2cm,bottom=2cm]{geometry}
\usepackage{xcolor}
\begin{document}
\section{Introduction}
The Rev Thomas Bayes was an 18th century cleric who was also a mathematician. His the- orem, known as Bayes’ Theorem and described in Section \ref{sec:thang}, was published posthumuously, and it is only thanks to LaPlace that it is so well known, as many other academics at the time failed to take the theorem seriously.
\section{Bayes’ Theorem}
\label{sec:thang}
Bayes’ Theorem is given in Equation 1, and simply states that the posterior of an event B given that event A has occurred is proportional to the prior for event B.\\
\begin{equation}
P(B|A) = \dfrac{P(A|B)P(B)}{P(A)}\\
\end{equation}
~~\textcolor{blue} {Equation} 1 is fundamental to the whole study and implementation of Bayesian belief networks.
\end{document}