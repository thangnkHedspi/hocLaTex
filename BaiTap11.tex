\documentclass[12pt,a4paper,twoside]{article}
\usepackage[utf8]{vietnam}
\usepackage{amsmath}
\usepackage{amsfonts}
\usepackage{amssymb}
\usepackage{graphicx}
\usepackage[left=2cm,right=2cm,top=2cm,bottom=2cm]{geometry}
\begin{document}
Hinh 9:\\
\begin{displaymath}
\left(
\begin{array}{cc}
5&3\\
2&6
\end{array}
\right) \\
= \left(
\begin{array}{cc}
10&6\\
4&12\\
\end{array}
\right)\\
\\Hinh 10:\\
\left(
\begin{array}{ccc|c}
a_11&\cdots & a_1n&b_1\\
 \vdots & \ddots & \vdots &\vdots \\
 a_n1 & \cdots & a_nn & b_n\\
\end{array}
\right)
\delta _{ij} = 
\begin{array}{cc}
1 & i = j\\
0 & i = j
\end{array}
\end{displaymath}
Có 3 loại môi trường để tạo ra các danh sách: itemsize(danh sách không đánh số), enumarate(danh sách đánh số) and description(danh sách đặt tên)
\end{document}